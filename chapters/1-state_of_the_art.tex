\chapter{\label{ch:1-state_of_the_art}State of the art}

Since the mid-1990s, museums, universities, institutes, and various other types of entities connected to time-based media art—and contemporary art in general—have been promoting initiatives aimed at conserving live media art, which has posed significant challenges to traditional conservation practices. These initiatives emerged from the urgent need to find new solutions to ensure the survival of this complex and ephemeral art form in the future. As a result, traditional conservation paradigms have been fundamentally reshaped, giving way to new conceptual frameworks that have laid the foundation for innovative strategies, including preservation methodologies and documentation models.\\
To date, hundreds of such initiatives have been launched—some of which are still active—and new ones continue to emerge. The problem of conserving live media art remains open (and perhaps it can never or should never be definitively closed). However, we are beginning to see shared principles and elements across these initiatives that warrant detailed analysis. Indeed, certain common factors have emerged which, while not officially standardised, have become widely accepted as fundamental to the conservation of this form of art.\\
This chapter aims to highlight the contributions of the many initiatives developed to date, with the goal of identifying the factors and tools currently at the core of new conservation strategies. The term ``initiative'' will be used here to focus on the activities that promote research into new conservation strategies. These activities can take various forms, such as projects, networks, working or research groups, and online or physical archives.\\
To conduct this study, we carried out a specific type of literature review—not based on traditional publications but rather on the activities carried out on the web by these initiatives. The web has become the primary tool that initiatives use to disseminate their research, though it comes with its own challenges. Therefore, this review could more accurately be called a ``web review.'' Through this review, we will examine both the quantitative and qualitative aspects of live media art conservation to define the state of the art—the current documentation, archiving, and conservation practices implemented by the entities involved.\\
It is important to note that this chapter will not delve into the theoretical and conceptual aspects of these practices, although some references will be made. The focus here is on analysing the practical aspects of conservation. The theoretical framework that underpins this topic will be explored in the next chapter, where we will outline the origins of these practical approaches as well as contemporary perspectives that have yet to be fully integrated into practice.\\
This order of presentation was chosen because, although the practical and theoretical research areas are closely interconnected, the latter has advanced significantly. Chapter~\ref{ch:3-mata-models_and_visual_language} will define the computational model presented in this text based on these theoretical advancements.

\section{Methodology}
The subject area and resources must be clearly defined at the foundation of any literature review. As Rowley and Slack \cite{rowley2004conducting} explain, ``\textit{A literature review distills the existing literature in a subject field; the objective of the literature review is to summarize the state of the art in that subject field.}” However, due to terminological confusion, identifying a central subject in the practice of live media art conservation is challenging. This confusion significantly impacts the review process and serves as a crucial analysis element in the review itself.\\
In order to avoid this terminological confusion in this thesis, we chose to use the term ``live media art''. However, some more common terms are ``time-based media'', ``media art,'' ``new media art,'' ``multimedia art,'' or more specific terms like ``digital art,'' ``installation art,'' or ``performance art.'' While these terms sometimes fully encompass the characteristics of current research (as described in the introduction), they are often only partially relevant. For example, ``performance art'' does not necessarily include technology-based works. Problems arise when terms are misused. For instance, ``multimedia'' is often applied in the digital art context but refers to the simultaneous use of multiple media (both analogue and digital) within an artwork \cite{friedman2023intermedia}. Similarly, ``time-based media art'' is frequently associated solely with video art, but as Laurenson \cite{laurenson2001developing} states, it includes any media that creates a time-based experience. This terminological inconsistency limits control during the review’s search phase, resulting in generic or inconsistent keywords and, consequently, unclear and noisy results.\\
Another significant challenge is the type of resources being studied. Since the goal is to analyse practical outcomes from initiatives such as projects, networks, and archives, traditional resources like papers and books are not the main focus. Instead, web-based resources—websites, web pages, and online posts—are more relevant. Scholars widely use these platforms to share findings and initiatives related to time-based media art conservation.\\
This shift to web-based resources has both advantages and disadvantages. On the positive side, websites are more open and accessible to a broad audience compared to articles or books. They also allow for the inclusion of diverse materials, particularly multimedia, which can be viewed directly on the site or downloaded. However, the use of web platforms has significant drawbacks, primarily related to maintenance. Maintaining a website requires ongoing effort, both in terms of work, as it needs regular updates and upgrades, and costs, such as paying for the domain. Consequently, websites are more susceptible to the obsolescence of information due to a lack of updates or expired domains—a problem also addressed later in this review.

\subsection{Review platforms}
The resources analysed regarding initiatives of live media art conservation were collected starting from two online platforms: \textit{Monoskop} and the \textit{International Network for the Conservation of Contemporary Art} (INCCA).
\begin{itemize}
    \item The \textit{Monoskop} website is a research platform for the arts, culture, and humanities. It presents wiki pages of contemporary themes and movements in art, culture, and society. Although \textit{Monosko} mainly focuses on the arts and artists, it is also a good search engine for live media art conservation initiatives. More importantly, it already presents a collection of resources on preserving and curating modern and contemporary art. The page (at the link \url{https://monoskop.org/Art/Care}) grew out of a collaborative effort within the \textit{New Approach in the Conservation of Contemporary Art} (NACCA) research network (last updated 24 November 2024). It is a handy starting point for a review of the state of the art.
    \item The \textit{International Network for the Conservation of Contemporary Art} (INCCA) website is a sharing platform for professionals connected to the conservation of modern and contemporary art (conservators, curators, scientists, registrars, archivists, art historians, artists, educators, students, etc.). The website is an ideal space to share the outputs of projects, organised initiatives, open calls, new archives, articles, and interviews within this field. Each INCCA member can share their posts, which are usually formatted with a presentation of the topic and one or more links to other web pages on which the subject is treated and presented in detail. The INCCA website showcases modern and contemporary art preservation initiatives and, thus, is a direct portal toward the websites of museums, archives, foundations, universities, and networks. 
\end{itemize}

\subsection{Review process}
\subsubsection*{Monoskop}
The review process started from \textit{Monoskop}’s resources collection webpage mentioned above. This page presents many lists of resources divided into several sections, such as ``Labs, initiatives, associations,'' ``Publications,'' ``Bulletins, newsletters,'' ``Films,'' ``Research projects, networks, consortiums,'' ``Events,'' ``Exhibitions,'' ``Software tools.'' Only the ``Labs, initiatives, associations'' and ``Research projects, networks, consortiums'' have been considered for the review, with a total of 130 entries. Because of the terminological confusion (many items were strictly related to analogue video preservation and digitalisation, and others were related to fine art), some entries of the lists were deleted, obtaining a total of 65 entries for the \textit{Monoskop} website.

\subsubsection*{INCCA}
The procedures for extracting resources from the INCCA search engine have been more elaborate. We initially collected posts related to conservation practices from the INCCA search engine. We used seven keywords: \textit{Document}, \textit{Archiv}, \textit{Preserv}, \textit{Conserv}, \textit{Restor}, \textit{Reactivat}, and \textit{Collect}
\footnote{Only the main part of the word is considered, taking into account both the noun, the verb and the gerund (e.g. \textit{Preserv} stands for \textit{preservation}, \textit{preserve} and \textit{preserving}.)
}.
 For each keyword, the search engine results from 6 (in the case of \textit{Reactivat}) to over 1500 posts (in the case of \textit{Conserv}), with a mean of 450 posts for each research and a total of 6304 collected posts. However, many posts overlap with each research.\\
For this reason, we developed a custom web scraper software in Python, which has been used to improve the website search and automatically remove overlaps and extract essential data (posts’ links, publication date, and authors)
\footnote{Web scraping software is a tool used to automatically extract data from websites. In our case, the software scanned all the results for each keyword used in the research, identified overlaps, and created a table with the necessary data extracted from the INCCA posts, including the post links, publication dates, and authors. It is a very simple tool that merely sped up the data extraction process.}. 
With this software, we collected 1092 posts. Each collected post was quickly analysed to verify the coherence with the main topic, deleting all the posts referring to other fields such as fine art, photography, and sculpture (as we have done with the \textit{Monoskop} list). We could also classify each post based on its function during this step. We defined five different classes: \textit{Entity} (presentation of an organisation, museum, project, and archive), \textit{Event} (invitation to a conference, summit, symposium, and exhibition), \textit{Publication} (when sharing and presenting books, articles, or theses), \textit{Program} (promotion of a University or organisation study program), \textit{Call} (call for participation to a conference or a publication), \textit{Other} (a miscellaneous of other and unclassified post). Every \textit{Entity} (organisation, university, museum, project, and archive) mentioned in the collected posts was further explored through Google search, allowing the collection of many other projects never mentioned in INCCA. Many entity's websites include sections usually named ``projects'' or ``research''; others have sections dedicated to external resources (e.g. as the ``other research project'' on the DOCAM website or the ``external resources'' page on the Matters in Media Art website which allow discovering external resources). The expanded research enables the collection of 76 total entities.\\
The complete collection results in 112 entries, 29 of which overlap between the \textit{Monoskop} and INCCA research procedures.\\

\newline
Each entry has been studied to obtain general information and specifications about the initiative’s objective, goals, and produced outcomes. The in-depth analysis starts from the ``home'' and ``about'' pages of the relative web resources (if present) and continues with reports and published articles. The research aims to extract specific information such as the typology of the initiative, used terminology, developed and utilised methodologies and strategies, and general ones regarding the period and geographical area of activity and founders. Some projects were particularly highlighted for their contribution and the relevance of their produced output.


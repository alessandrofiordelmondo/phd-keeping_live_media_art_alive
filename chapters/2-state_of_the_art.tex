\chapter{State of the art}

\section{Methodology}
The subject area and resources must be clearly defined at the foundation of any literature review. As Rowley and Slack \cite{rowley2004conducting} explain, “\textit{A literature review distils the existing literature in a subject field; the objective of the literature review is to summarize the state of the art in that subject field.}” However, due to terminological confusion, identifying a central subject in the practice of time-based media art preservation is challenging. This confusion significantly impacts the review process and serves as a crucial element of analysis in the review itself.\\
The term “time-based media art” is not universally accepted or consistently used in the field. It is often replaced with other terms such as “media art,” “new media art,” “multimedia art,” or more specific terms like “digital art,” “installation art,” or “performance art.” While these terms sometimes fully encompass the characteristics of current research (as described in the introduction), they are often only partially relevant. For example, “performance art” does not necessarily include technology-based works. Problems arise when terms are misused. For instance, “multimedia” is often applied in the digital art context but actually refers to the simultaneous use of multiple media (both analogue and digital) within an artwork \cite{friedman2023intermedia}. Similarly, “time-based media art” is frequently associated solely with video art, but as Laurenson  \cite{laurenson2001developing} states, it includes any media that creates a time-based experience. This terminological inconsistency limits control during the review’s search phase, resulting in generic or inconsistent keywords and, consequently, unclear and noisy results.\\
Another significant challenge is the type of resources being studied. Since the goal is to analyse the practical outcomes from initiatives such as projects, networks, and archives, traditional resources like papers and books are not the main focus. Instead, web-based resources—websites, web pages, and online posts—are more relevant. Scholars widely use these platforms to share findings and initiatives related to time-based media art conservation.
This shift to web-based resources has both advantages and disadvantages. On the positive side, websites are more open and accessible to a broad audience compared to articles or books. They also allow for the inclusion of diverse materials, particularly multimedia, which can be viewed directly on the site or downloaded. However, the use of web platforms has significant drawbacks, primarily related to maintenance. Maintaining a website requires ongoing effort, both in terms of work, as it needs regular updates and upgrades, and costs, such as paying for the domain. Consequently, websites are more susceptible to the obsolescence of information due to a lack of updates or expired domains—a problem also addressed later in this review.

\section{Review platform}

%!TEX root = ../main.tex

\begin{abstract}

\section*{EN}
This text explores the new paradigms and conservation practices for live media art—a type of art that combines real-time, interactive, or performative elements (live) with various media and technological devices. These practices–which have developed since the 1990s–particularly challenge traditional principles of conservation, such as originality and authenticity.\\
The main objective of this research was to develop and apply a meta-model for conservation: the \textit{Multilevel Dynamic Conservation} (MDC) model. This model aims to surpass the very definition of a model (as suggested by the prefix ``meta'') and address the challenges of standardising conservation practices. It is designed to support the description and development of multiple conservation models necessary for handling the heterogeneity of live media works and responding to the diversity of entities involved in their conservation. To support this model, an reactivation workflow (the CATTA workflow) and a Modular Instruction Framework (MIF) for documenting technological devices and their processes were also defined. \\
The outcomes of this research were developed through a review of online resources, literature analysis, and, especially, a series of diverse case studies. These case studies explored not only the documentation and reactivation of artworks but also the role of artists in technological development, the interdisciplinary collaboration systems that support the artwork’s creation, the technological basis of performances and installations, and the role of audiences in interactive ecosystems.\\
The application of the meta-model, reactivation process, and modular instruction in the case studies demonstrated strong adaptability, enabling support for works in various performative, installation, and research contexts. In particular, the MDC model proved to be highly flexible in multiple circumstances and across different computational systems, such as documenting an artwork using the graphical database Neo4j, supporting collaboration in an artistic interdisciplinary research project through the development platform GitHub, and creating an “actor's archive”—personal documentation of a performer—by locally managing the documentation on a computer (in our case, Finder on macOS).\\
This study particularly highlights, on the one hand, the complexity of the new conservation practices for live media art, which must adapt and reformulate according to each artwork; on the other hand, it emphasises the fundamental principles of the new conservation paradigms upon which the developed systems are based. The new conservation paradigms do not aim to freeze the artwork in its original state but rather to allow it to evolve and grow. The artwork itself is seen as a living organism or process that interacts with the exhibition  ecosystem. This ecosystem includes media and technological devices, spaces, humans as both audience and performers, and the socio-cultural moment, which is always in constant transformation. Thus, the artwork, like a living organism, needs to adapt, grow, take shape, and develop its identity in relation to the evolving exhibition ecosystems. Therefore, the purpose of conservation, as well as the systems presented in this text, is to keep live media art alive, supporting its development and becoming.
\newpage

\section*{IT}
Questo testo esplora i nuovi paradigmi e pratiche di conservazione della Live Medi Art –ossia l’arte che combina elementi in tempo reale, interattivi o performativi (live), con vari dispositivi mediali e tecnologici. Queste pratiche—sviluppatisi a partire dagli anni ‘90—stravolgono particolarmente i principi tradizionali di conservazione, come i concetti di originalità e autenticità.\\
L’obiettivo principale di questo progetto di ricerca è stato quello di sviluppare e applicare un meta-modello per la conservazione, il \textit{Multilevel Dynamic Conservation} (MDC) Model. Questo modello si propone di andare oltre la semplice definizione di modello (come suggerisce il prefisso ``meta'') e di affrontare le limitazioni legate all’impossibilità di standardizzare le pratiche conservative. È progettato per supportare la descrizione e lo sviluppo di modelli di conservazione multipli, necessari per gestire l’eterogeneità delle opere di live media art e rispondere alla diversità degli enti coinvolti nella loro conservazione. A sostegno di questo modello, sono stati definiti un approccio per la riattivazione (il CATTA workflow) e un linguaggio visivo per documentare i dispositivi tecnologici e i loro processi (il \textit{Modular Instruction Framework}).\\
I risultati di questa ricerca sono stati sviluppati attraverso una review delle risorse online, un’analisi della letteratura e, in particolare, una serie di studi di caso. Questi casi, molto diversi tra di loro, hanno avuto lo scopo di esplorare la documentazione e la riattivazione delle opere, il ruolo degli artisti nello sviluppo tecnologico, i sistemi di collaborazione interdisciplinare che supportano la creazione delle opere, le basi tecnologiche delle performance e delle installazioni, e il ruolo del pubblico negli ecosistemi interattivi.
L’applicazione del meta-modello, del processo di riattivazione e del linguaggio visivo nei casi studio ha dimostrato una forte adattabilità, consentendo di supportare le opere in una varietà di contesti performativi, installativi e di ricerca. In particolare, il modello MDC si è rivelato altamente flessibile in diverse circostanze e attraverso vari sistemi computazionali, come documentare un’opera utilizzando il database grafico Neo4j; supportare la collaborazione in progetti di ricerca artistica interdisciplinare tramite la piattaforma di sviluppo GitHub; e creare un “archivio d’attore” —e.g. documentazione personale di un performer — gestendo localmente la documentazione su un computer (nel nostro caso, attraverso il Finder dei sistemi macOS).\\
Questo studio evidenzia, da un lato, la complessità delle nuove pratiche di conservazione per live media art, che devono adattarsi e riformularsi in base a ciascuna opera; dall’altro, sottolinea i principi fondamentali dei nuovi paradigmi di conservazione su cui si basano i modelli. I nuovi paradigmi di conservazione e i modelli sviluppati non mirano a cristallizzare l’opera nel suo stato originale, ma piuttosto a consentirne l’evoluzione e la crescita. L’opera stessa viene vista come un organismo vivente o un processo che interagisce con l’ecosistema della installazione o della performance. Questo ecosistema include media e dispositivi tecnologici, spazi, esseri umani sia come pubblico sia come performer e, soprattutto, il momento socio-culturale, sempre in costante trasformazione. Pertanto, l’opera, come un organismo vivente, ha bisogno di adattarsi, crescere, prendere forma e sviluppare la propria identità in relazione agli ecosistemi espositivi in evoluzione. Di conseguenza, lo scopo della conservazione, così come dei modelli presentati in questo testo, è mantenere viva la Live Media Art, sostenendone lo sviluppo e il divenire.
\end{abstract}

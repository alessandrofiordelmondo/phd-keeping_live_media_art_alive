\chapter{\label{ch:0-methodology}Methodology}
This research project aims to develop a computational model for documenting, reactivating, and conserving live media artworks. We use an interdisciplinary approach that combines expertise from the artistic-creative, humanities, and technological fields. Since this project involves designing a model to create both descriptive and applied knowledge for a specific research context, we decided to use a Design Science Research (DSR) methodology.
\begin{quote}
Simply stated, DSR seeks to enhance technology and science knowledge bases via the creation of innovative artifacts that solve problems and improve the environment in which they are instantiated. The results of DSR include both the newly designed artifacts and design knowledge (DK) that provides a fuller understanding via design theories of why the artifacts enhance (or disrupt) the relevant application contexts. \cite{brocke2020introduction}
\end{quote}
We apply DSR drawing on \cite{brocke2020introduction}, which incorporates the DSR research framework proposed in \cite{hevner2004design}, the DSR process proposed in \cite{peffers2007design}, and the DSR evaluations proposed in \cite{sonnenberg2012evaluations}.\\
Other tools we used for the research include an experimental web-based literature review and case studies, the latter being our main instrument for empirical inquiry.

\section{Research Framework}
According to \cite{hevner2004design} and \cite{brocke2020introduction}, DSR aims to create innovative solutions to a specific research problem. These solutions build on existing knowledge bases (KB)—including theories, methods, and tools—and address \textit{needs} within an \textit{Environment} (i.e., the research problem).\\
In this study, the \textit{Environment} consists of institutional and private organizations, as well as individuals, who manage cultural heritage and \textit{need} to preserve live media artworks in their complex experiential ecosystem.\\
We established the KB by conducting 1) a web-based review (Chapter~\ref{ch:1-state_of_the_art}), focusing on practical outcomes like tools and models developed in \textit{Environment}-related initiatives, and 2) a non-systematic literature review (Chapter~\ref{ch:2-new_conservation_paradigms}), focusing on theoretical and conceptual aspects of conservation.\\
The design of the systems—i.e., the contributions of this research—was refined through several iterations guided by the KB and by case studies, which allowed us to evaluate and optimize each design for our \textit{Environment}.

\section{Research process}
According to the six-step DSR process outlined in \cite{peffers2007design} and \cite{brocke2020introduction}, we conducted our study using the following steps:
\begin{enumerate}
    \item \textit{Problem Identification and Motivatio}. We conducted reviews to establish the state of the art (or state of the problem) and define the research problem in detail (the \textit{ecological turn} and the non-standardisable nature of conservation in Chapter~\ref{ch:1-state_of_the_art}, \ref{ch:2-new_conservation_paradigms}, and \ref{ch:3-mdc_model-reactivation_workflow-instruction_template}). We recognized the need for an innovative, abstract model rather than a strictly standardised approach (Chapter~\ref{ch:3-mdc_model-reactivation_workflow-instruction_template}).
    \item \textit{Define Objectives for a Solution}. We determined the goals of our solution—namely a meta-model, a reactivation workflow, and an instruction template—based on the need for broad applicability and flexibility (Chapter~\ref{ch:3-mdc_model-reactivation_workflow-instruction_template} and \ref{ch:4-madc_model_application}).
    \item \textit{Design and Development}. We defined the functions and architecture of these systems (MDC, CATTA, and MIF). We refined them through feedback from the KB and the case studies.
    \item \textit{Demonstration}. We tested these systems in several case studies, applying them to real-world scenarios relevant to our \textit{Environment} (Appendix~\ref{ax:a-michele_sambin_videoloop}, \ref{ax:b-hybrid_reactivation_il_caos_delle_sfere}, \ref{ax:c-the_score_in_live_electronics_music}, and \ref{ax:d-sustainability_and_longevity_of_nimes}).
    \item \textit{Evaluation}. We assessed the systems’ effectiveness in addressing the research problem, revisiting earlier phases to make improvements when needed.
    \item \textit{Communication}. We shared our results with both the scientific community and relevant stakeholders \cite{fiordelmondo2023multilevel, fiordelmondo2023toward, fiordelmondo2024nime, fiordelmondo2024reactivating}. This dissemination yielded valuable feedback for revisiting our designs and solutions.
\end{enumerate}

\section{Case study}
Case studies served as our main tool for empirical inquiry, system design, and evaluation. They offered real-life contexts in which the phenomenon (live media art conservation) and its \textit{Environment} overlapped \cite{yin2009case}. This allowed us to refine our artifacts and expand our KB.\\
We conducted 11 case studies in total, two of which (Appendices~\ref{ax:a-michele_sambin_videoloop} and \ref{ax:b-hybrid_reactivation_il_caos_delle_sfere}) are extensive and complex, while the other nine (Appendix~\ref{ax:c-the_score_in_live_electronics_music}, and \ref{ax:d-sustainability_and_longevity_of_nimes}) address more targeted topics that supported the systems’ development. Other case studies were carried out but are not included here due to their lower relevance to the overall thesis results. However, they also contributed to the design and development of the systems\footnote{The main omitted case studies focused on studying the technologies used in creative contexts and the documentation process during the creation phase. They include: 1) The development, documentation, and reactivation of \textit{Disallineato}, an interactive multimedia work (presented in Rome and Milan in 2023); The development and documentation of \textit{Come Terra}, an installation based on environmental sensor data (presented in Padua in 2023); The development and documentation of \textit{AI ludivig van…?}, a performance and study on musical creativity using artificial intelligence (presented in Bolzano in 2023).}.\\
In all cases, we documented and reactivated creative works using the solutions we designed and examined how well these solutions functioned for conservation, based on criteria drawn from the KB.

\section{Evaluation}
Following \cite{sonnenberg2012evaluations} and \cite{brocke2020introduction}, we evaluated our designs throughout the DSR phases and distinguishing between \textit{Ex Ante} and \textit{Ex Post} evaluations. \textit{Ex Ante} evaluation occurs before implementing solutions. We used criteria like importance, novelty, applicability, simplicity, and consistency, comparing the proposed design with existing KB and problems. \textit{Ex Post} evaluation occurs after implementing solutions in case studies. We assessed fidelity, ease of use, effectiveness, efficiency, and robustness, and compared the outcomes with the KB to measure how well our systems addressed the research problem. \\  
The case study also enabled evaluation from the very people involved (e.g., artists), also through interviews (for instance, in the case studies reported in Appendix~\ref{ax:a-michele_sambin_videoloop} and Appendix~\ref{ax:c-the_score_in_live_electronics_music}).\\
Each chapter or appendix contains a summary or discussion of the relevant evaluation results, which are synthesized in the final conclusions.

\section{Design Knowledge}
DSR results include not only the artifacts (the meta-model, workflow, and instruction template) but also design knowledge (DK)—the insights generated through design and evaluation in the defined \textit{Environment} \cite{brocke2020introduction}. In other words, DK arise when the problem space (the \textit{Environment}’s \textit{needs}) is connected with the design space (the artifacts) and vice versa, through evaluation. This approach aims to produce artifacts with both \textit{descriptive $\Omega$-knowledge} (improving our understanding of the involved phenomena) and \textit{application $\lambda$-knowledge} (new, practical insights that foster future development) \cite{gregor2013positioning, winter2013restructuring}.

\section{Limitation}
The main limitations come from the \textit{Environment}, which is extremely broad, heterogeneous, and interdisciplinary, and from the participatory approach required by the case studies. Although we initially planned to restrict our research scope to specific live media art production and exhibition contexts, this was not possible due to a lack of case studies. Besides those presented in the appendices (and those omitted), other case studies were started but did not continue or conclude, mainly because the people involved were not always available. Thus, we could not methodically choose precise types of case studies. Instead, we worked more heterogeneously, reflecting the \textit{Environment}’s complexity.\\
Other problems related to the \textit{Environment} include terminology issues and a wide variety of communication and dissemination methods. These challenges led us to create the experimental web-based review in Chapter~\ref{ch:1-state_of_the_art}, which naturally introduced biases and may limit the broader relevance of the results compared to a more formal systematic literature review. Chapter~\ref{ch:1-state_of_the_art} discusses these topics in detail.

\section{Summary}
In this chapter, we described the Design Science Research (DSR) methodology used in our study, along with the main tools—web-based reviews and case studies—our evaluation criteria, and the limitations. 
%$Figure X summarizes the research methodology.


%!TEX root = ../main.tex

\chapter*{\label{ch:introduction}Introduction}
\markboth{Introduction}{}
\addcontentsline{toc}{chapter}{Introduction} 
% NOTE: Introduction and Conclusions are special chapters: they are not
%       marked by a section number, so we need to add it manually to the
%       ToC, as done above.

The term \textit{live media art} refers to a type of art that combines real-time, interactive, or performative elements (live) with various media devices and electronic technology. This field includes many art forms that have developed and become recognised throughout the last century, especially with the rise of electronic and digital technologies. Live media art includes installation and performance art, which are the main ways these works are presented. It also covers video art that involves interaction, whether from the audience or performers. Additionally, interactive intermedia and multimedia art—where different media and artistic practices are blended—are part of live media art. Artwork based on new audiovisual experiences like virtual and augmented reality are also included. Computer, digital, software, and internet art all fall under live media art. Live media art is now considered part of contemporary art, as works of this genre have already entered the mainstream of art and can be found in many museums, galleries and exhibitions worldwide. The term live media art is not widely used in literature. It is used here to avoid entering the terminological confusion and conflicts of other more familiar terms, such as Media Art, New Media Art, Time-based Media Art, and Electronic Art.\\
Moreover, live media art emphasises the ``living'' aspect of this art. Not just ``living'' because it happens live, but because the artwork acts as a ``living being.'' Instead of being fixed, the artwork acts as an agent that activates and connects a network of entities from various interconnected ecosystems. These ecosystems vary in complexity and include media devices and technologies that play key roles in allowing the artwork to interact. They also include institutions, diverse spaces (such as museums, parks, gardens, urban landscapes, and even virtual spaces) and humans as audiences, performers, collaborators, curators, and conservators. Most importantly, the socio-cultural context of the artwork plays a crucial role. The entire network both depends on this context and actively shapes it. Through its interaction with media and technological devices, the artwork responds to the conditions that make the art experience possible within its socio-cultural setting. It establishes a vital connection with the present and shapes how the experience is understood. However, the rapid changes in media and technological devices, which affect the socio-cultural aspects and thus the institutional and economic landscape, lead to quick changes in the artwork’s ecosystem. This continuous becoming often distorts the original intended experience. As a result, the artwork becomes ephemeral more than ever before—not only during its exhibition but also in its existence after the exhibition ends. Live media artworks actively engage with the present and may die along with it.\\
These unique features of live media art raise the main question of this research: \textit{How can live media art be kept alive?}\\
The complex nature of live media art requires rethinking and redefining traditional conservation methods and practices. Conservation, including preservation and restoration, has traditionally relied on the artwork’s authenticity, determined by its physical and material integrity, which allows for its historical and authoritative identification. First of all, we must recognise that live media artworks are heterogeneous from one another, not only in terms of the materials and technologies used but also in terms of the processes involved and the types of experiences they offer. The technological and material composition is varied, consisting of a collection of components, hardware, and software. The material components are often consumable, and the technologies are prone to becoming obsolete. This problem not only leads to the premature ageing of technology in a material and practical sense but also, more importantly, causes the experiences produced and the surrounding knowledge to age and fade away. This ageing is significant and critical because live media art, being ``live,'' is defined by the process- and time-based experiences created by those specific assemblies of components and technologies. Furthermore, this experience is also determined by the exhibition's surrounding space, which, depending on the artwork, can establish a stronger connection. We must also reevaluate authorship. Indeed, live media artworks are often the product of a series of scattered contributions, involving people who know each other and those who do not, through the collaboration and participation of multiple artists, technicians, curators, performers, and audiences. Therefore, in this scenario, it is impossible to consider the main factors of traditional conservation, such as physical and material integrity and authorship, as they are entirely distorted here.\\ 
From this issue, in the late 1980s, a specific interdisciplinary research field emerged. The interest - and especially the need - to conserve live media art began to grow in those years as a natural consequence of the institutionalisation of these new forms of artistic creation. During this period, the number of art events focusing on live media art and institutions dedicated to it increased. Key events of the 1980s include the 1985 \textit{Les Immatériaux} exhibition at the Centre Pompidou in Paris, which was significant not just for the artworks displayed but especially for transforming the exhibition into an interactive installation. Another important event was the 1986 Venice Biennale, particularly the ``Tecnologia e Informatica'' section, which featured the \textit{Planetary Network} (a three-week networking project coordinated by Roy Ascott) and multimedia and interactive works by artists such as Rokebay, Leeson, and Eno. Above all, 1989 was a pivotal year for this type of art, marked by the founding of the Zentrum für Kunst und Medientechnologie (ZKM) in Karlsruhe, Germany. This institution played a crucial role in the institutionalisation, production, and later the conservation of live media art \cite{quaranta2013beyond}. From this point onward, growth became exponential, with the promotion of new events, the establishment of institutions, and the creation of private and museum collections. Immediately as a result, starting in the mid-1990s, the organisations responsible for institutionalising live media art began promoting the first initiatives aimed at developing new conservation strategies. One of the first and most significant initiatives in the field was \textit{Modern Art: Who Cares?} launched in 1995 by the Stichting Behoud Moderne Kunst (SBMK) in the Netherlands. This three-year research project addressed the theoretical and practical conservation of ten non-traditional works, including Mario Merz's \textit{Città irreale}, Jean Tinguely's \textit{Gismo}, and Pietro Gilardi's \textit{Still Life of Watermelons} \cite{van2024conservation}. Subsequently, –in parallel with the institutionalisation of this artistic genre that was still in the process of being established– new initiatives emerged to study and formalise new conservation strategies. These initiatives included new research projects, conferences, working groups, networks, and more. This research area also began to expand beyond museum institutions, conservators, and heritage organisations in general, with universities and academies also becoming involved. In fact, the fundamental question of how to conserve these works began to raise issues in art history and philosophy: What constitutes an artistic work in any given case, and what is the role of institutions–like museums–in maintaining it? The conservation of contemporary art has become a distinct field of research \cite{laurenson2022bridging}. Besides sharing projects through web platforms, which museums and other institutions often use to distribute their research, we also see academic articles, special issues, theses, and doctoral programs. A recent example is the Marie Sadowska-Curie Innovative Training \textit{Network on New Approaches in the Conservation of Contemporary Art} (NACCA), a doctoral program held from 2015 to 2019 that included 15 interdisciplinary research projects.\\
To date, new conservation paradigms have been established. The idea of original artwork is replaced by the concept of a \textit{variable} artwork \cite{depocas2003variable} that evolves to survive the constant changes in its experiential ecosystem. The importance of fixed authenticity is thus replaced by an \textit{allographic} system (as proposed in \cite{laurenson2004management} according to the distinction \textit{autographic-allographic} in \cite{goodman1968languages}), where authenticity becomes dynamic \cite{innocenti2012bridging, innocenti2012rethinking}, and the artwork’s identity constantly evolves. Physical materiality and integrity are replaced by \textit{stratigraphy of documentation} \cite{holling2016aesthetics}, which includes informational sheets, interviews, and statements, as well as scripts, scores, and instructions that define \textit{work-defined properties} \cite{laurenson2006authenticity} allowing the artwork to be reactivated. This reconstruction happens according to the artwork’s \textit{changeability} \cite{holling2017paik} through philological reconstructions and, more importantly, through technological migrations, emulations, and reinterpretations. When an artwork enters a museum, it does not remain in its original state. Instead, it enters a \textit{state of infancy} \cite{phillips2015reporting}, where its identity will be \textit{under construction} \cite{dekker2022documentation} and under formation \cite{phillips2015reporting} through new conservation practices. The artwork becomes an \textit{epistemic object} whose meanings continually emerge through their indefinite \textit{unfolding} \cite{laurenson2016practices}. The comparison of live media art to ``life'' becomes even more evident in the act of its conservation. It is no longer about simply preserving the artwork but about securing the conditions that allow it to become \cite{butler2018my, castriota2019authenticity}, evolve, and mature like a living being according to the changes in its ecosystem. This comparison is especially emphasised in the latest developments in live media art conservation, known as the \textit{practical} or \textit{ecological turn} \cite{van2023theories}. Conservation must now consider the \textit{variability} and \textit{changeability} of the artwork in relation to technological, institutional, creative, and general exhibition ecosystems. This aspect is essential because the artwork’s \textit{biography} \cite{van2011reflections} and its ontologies depend on these ecosystems.\\
In response to these new conservation theories, museums and institutions have started developing practical implementation tools. The main contributions are original documentation models designed to capture the complexity of the artworks and, most importantly, their variability.\\
Today, this field of research, now thirty years old, has gained strong stability both conceptually and theoretically, as well as practically. However, we are still far from having an absolute definition of the new paradigms and even further from standardising conservation practices. Especially for standardisation, the path to achieving a fixed approach is still very long. The models and practices developed by institutions are still closely tied to the history, subject, scale, and structure of those institutions \cite{wielocha2024collections}, making them difficult to apply in other contexts. However, in this situation, pursuing standardisation might be misguided. Instead, it might be better to remain flexible and diverse, just like the artworks at the centre of this research field.\\
From this premise, the main contribution of this research is defined: the \textit{Multilevel Dynamic Conservation} (MDC) model. This model is developed based on conservation paradigms and takes into account the complexity of the artwork (and creative works in general) and its experiential ecosystem, achieved through its hierarchical multilevel structure, as well as its variability and forming identity, by defining iteration series and the property we call \textit{multiple belongingness}. However, the most important aspect is that this model is intended to be used as a meta-model. The prefix ``meta'' indicates that the model is an abstraction of conservation models and the new conservation paradigms. It serves to define the properties with which the artwork should be understood in the context of its conservation. Therefore, this meta-model can be used both to describe existing models and, more importantly, to create new ones. The strength of such a model lies in its ability to remain abstract, allowing for a variety of applications that can occur at different levels of complexity. In fact, this model considers the \textit{ecological turn} from an application perspective, making it applicable at institutional, collaborative, and personal levels based on the available resources and tools. Thus, the MDC model presents itself as an alternative and opposite to standardisation, promoting a controlled variety of applications. This means having several applications that follow a set of guidelines and can marge into the properties used.

This text will present the MDC model and its use in various application cases. At the same time, we will introduce two other contributions: a model to guide the reactivation process called CATTA (\textit{Collection}, \textit{Assessment}, \textit{Transcription}, \textit{Transmission}, and \textit{Archiving}) reactivation workflow and a template for creating the technical instructions of an artwork, called the \textit{Modular Instruction Framework} (MIF). The MIF incorporates a visual language for documenting technological devices and their processes. It is an adaptation of a language introduced by Marije Baalman in the book \textit{Composing Interactions} \cite{baalman2022composing} for describing digital musical instruments (DMIs) and their use in multimedia performances. For this reason, we named our implementation of \textit{Baalman’s Visual Language} as BVL.\\
The contributions of this research have been developed through compiling a web-based review, analysing and summarising new conservation paradigms, and, most importantly, through a series of diverse case studies. These case studies were used to study: the documentation and reactivation of the artwork; the role of artists in technological development processes; the processes and systems of interdisciplinary collaboration that support the artwork’s development; the technological basis of performances and installations; the role of audiences in the interaction; and of course the application of the MDC models, the CATTA reactivation workflow and the instruction template with the BVL.\\

\section*{Thesis structure}
This work is divided into two main parts: 1) the body of the research, which goes from the state of the art to the presentation and application of the models, and 2) the case studies. For convenience, the case studies are presented as appendices at the end of this text. However, it is important to note that they should not be considered as a mere supplement or additional resource, but rather as an integral part of the work. The case studies were essential tools for deriving and applying the concepts and systems studied and developed in this thesis. In the main text, the case studies will be used to illustrate specific conceptual and practical points, and, of course, the studies will refer back to both the theoretical and practical content of the main text. Therefore, the two parts should be seen as equal and strictly connected.\\
\newline
In the first part of the text, we present the state of the art, the fundamental concepts of new conservation practices, and the main contributions of this research: the MDC model, the CATTA reactivation workflow, and MIF along with the BVL language.\\
Chapter~\ref{ch:0-methodology} explains the research methodology used in this study, which is based on the Design Science Research (DSR) approach, with a special focus on case studies. It defines the scope and limitations of this study.\\
Chapter~\ref{ch:1-state_of_the_art} presents the state of the art of the initiatives developed from 1995 to today within this research field. As we will see, this chapter is based on an experimental approach: the ``web-based review.'' Since much of the dissemination of research in this context has happened (and still happens) through web platforms, our research began by exploring two main platforms: \textit{Monoskop} and the \textit{International Network for the Conservation of Contemporary Art} (INCCA). Despite certain limitations and biases, we collected and studied 113 initiatives promoted by over 300 institutions, obtaining quantitative and qualitative data. This chapter is essential for gaining an international overview of this research area from a temporal, geographical, and institutional perspective, as well as from a terminological, conceptual, and especially practical one. Indeed, one of the main goals of this chapter is to study how conservation practices are implemented within institutions—such as the development of documentation models, preservation strategies, and more.\\
Chapter~\ref{ch:2-new_conservation_paradigms} provides a deeper theoretical and conceptual look into the new conservation paradigms. Here, we focus more on academic literature, particularly the theories that guide the initiatives and applications presented in the previous chapter, as well as others that still remain substantially in the theoretical domain. Although this chapter also deals with the state of the art, its main goal is to use this literature to form the theoretical structure behind the principal contributions. Indeed, we can view this chapter as the first contribution in which we define: 1) documentation and its functions, 2) the relationship between authenticity and identity in the context of live media art, 3) conservation as a loop between documentation and reactivation, and 4) reactivation itself as a process of transcription and transmission. We also introduce the idea of the artwork as a living being and \textit{actant} within a complex network of interactions, alongside the \textit{ecological turn} in new conservation practices.\\
Chapter~\ref{ch:3-mdc_model-reactivation_workflow-instruction_template} introduces the main contributions of this research: 1) the \textit{Multilevel Dynamic Conservation} (MDC) model as a meta-model with its main properties; 2) the CATTA (\textit{Collection}, \textit{Assessment}, \textit{Transcription}, \textit{Transmission}, and \textit{Archiving}) workflow as a tool to guide the reactivation process; and 3) the \textit{Modular Instruction Framework} (MIF) a modular template for compiling instructions that also includes and extends \textit{Baalman’s Visual Language} (BVL). In particular, we will see how the first two contributions represent a structured abstraction of the models discussed in Chapter~\ref{ch:1-state_of_the_art} and the conservation paradigms studied in Chapter~\ref{ch:2-new_conservation_paradigms}.\\
Chapter~\ref{ch:4-madc_model_application} shows the application of the MDC model in three different scenarios. First, we define the metadata structure of the model, which we then apply in representative use cases that relate to the included case studies: 1) a complex application using Neo4j, a graph database; 2) a collaborative application using the GitHub development platform; and 3) a personal e simplified application for creating an ``author archive,'' using a simple folder management system (in our example, Finder). This chapter will briefly discuss how the BVL can be adapted and used with Diagram.net, a free, online, cross-platform tool.\\
\newline
The second part—the case studies—consists of four appendices. These studies are presented as short standalone essays that follow a nearly fixed textual model: an introduction to the study and its objectives (\textit{Objectives}), an introduction to the artworks or creative works on which the study is based (\textit{Introduction}), a reactivation process carried out according to the CATTA workflow (\textit{Reactivation}), the compilation of instructions and application of the BVL (\textit{Data Compilation}), and finally a discussion about the study itself, its relation to the conservation paradigms and a summarisation of the results (\textit{Results and discussion}).\\
Appendix~\ref{ax:a-michele_sambin_videoloop} covers a case study on reactivating a performative video artwork by Michele Sambin. This study is likely the most extensive and most extended study conducted during this research, as the reactivation of the \textit{videoloop}, specifically the performance \textit{Il tempo consuma} created in the late 1970s, is still being presented at various Italian festivals (most recently on October 20, 2024). This work involved migrating the analogue \textit{videoloop} system to the digital domain and reactivating the performance, which was dormant for 42 years. The study involved close collaboration with the artist, experimenting with digital technologies to reactivate the system, and reinterpreting the performance.\\
Appendix~\ref{ax:b-hybrid_reactivation_il_caos_delle_sfere} focuses on reactivating a 1999 musical installation by Carlo De Pirro, \textit{Il caos delle sfere}. In this installation, the audience is invited to play with a 1990s pinball machine, connected through a hardware/software system that uses the game’s output to generate musical sequences on a Disklavier (a piano with automated keys). This case study was both complex and essential for developing the MDC model and testing conservation strategies. Unlike the first study, we did not have the direct collaboration of the artist (who sadly passed away in 2008). Still, some of his collaborators who worked on the hardware and software in 1999 provided partial support. The work has been inactive since 2012 and has substantial hardware issues, so it underwent a detailed analysis (using a software archaeology approach). It led to a \textit{hybrid} reactivation where the hardware/software part was rebuilt with new devices and software (digital-to-digital migration), while the interaction and feedback part was carefully reconstructed using the original pinball and a Disklavier. This study proved essential for observing how audiences react to an installation from over 20 years ago, mainly reactivated faithfully.\\
Appendix~\ref{ax:c-the_score_in_live_electronics_music} presents a broader study involving six live electronics pieces composed between the 1980s and 1990s: four by Luigi Nono –\textit{Das Atmende Klarsein} (1981), \textit{Quando stanno morendo. Diario Polacco n.2} (1982), \textit{Guai ai gelidi mostri} (1983), \textit{A Pierre. Dell’azzurro silenzio inquietum} (1985)– and two by Salvatore Sciarrino –\textit{Perseo e Andromeda} (1991) and \textit{Cantare con silenzio} (1999). These pieces were reactivated and performed under the guidance of the original technicians and performers, and for Sciarrino’s pieces, even with the guidance of the composer himself. The main goal of this case study was to examine the musical score in electronic music, especially regarding technological instructions and processes. Unlike many live media artworks, electronic music routinely involves the compilation of a score, the documentation of technologies, and the periodic adaptation to the evolving technological ecosystems. As discussed in Chapter~\ref{ch:3-mdc_model-reactivation_workflow-instruction_template} of the first part of this text, this case study formed the basis for developing the MIF and adapting the BVL. The structure of this case study is slightly different in that a \textit{Score} section is added, providing an analysis of the six pieces’ scores.\\
Appendix~\ref{ax:d-sustainability_and_longevity_of_nimes} explores how the MDC model, the CATTA reactivation workflow, and the MIF with the BVL are applied in a collaborative and development-oriented context, focusing on digital musical instruments (DMIs) and the \textit{New Interfaces for Musical Expression} (NIME). It involves the documentation, archiving, and reactivation of the \textit{Soundrise} multimedia application, as well as the documentation and archiving of two DMIs created by Patricia Cadavid, \textit{Electronic\_Khipu\_} and \textit{Kanchay\_Yupana//}. The systems developed in this research were used to promote the longevity and sustainability of DMIs in the NIME context. Although this study does not deal strictly with artworks, but rather with creative works, it was crucial for examining how to maintain works in a collaborative (and sometimes large-scale) setting—such as Patricia’s DMIs, which span three continents (South America, Europe, and Asia). It was also particularly important to test the MDC model on a development platform like GitHub.

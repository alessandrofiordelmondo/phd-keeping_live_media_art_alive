%!TEX root = ../main.tex

\chapter*{\label{ch:conclusions}Conclusions}
\markboth{Conclusions}{}
\addcontentsline{toc}{chapter}{Conclusions} 
% NOTE: Introduction and Conclusions are special chapters: they are not
%       marked by a section number, so we need to add it manually to the
%       ToC, as done above.

The main research question posed at the beginning of this thesis was: \textit{How can live media art be kept alive?} This question underpins the conceptual framework developed throughout the research and serves as the foundation for its main contributions: the model for documenting, reactivating, and conserving contemporary artistic practices.\\
As discussed in Chapters~\ref{ch:1-state_of_the_art} and~\ref{ch:2-new_conservation_paradigms}, conservation has shifted from preserving an artwork's ``original'', ``authentic'' state to safeguarding the conditions that allow the artwork to unfold. Conservation now involves keeping the artwork ``alive'' through its various reactivations, allowing it to form and reform within the ever-changing dynamics of its experiential ecosystem. Artworks are no longer static objects bound to an absolute idea of experience but are instead defined by those experiences generated within specific socio-cultural contexts, making them as ephemeral as those contexts. Contemporary artistic practices often utilise media and technology to explore the expressive possibilities of these tools, detaching them from their practical functions. Such practices highlight the convergence of aesthetic and social functions within a single expression, reflecting on how humans live within, think about, and interact with their technological ecosystems.\\
Case studies and examples in the text demonstrate how contemporary art inquiries the expressive potential of technology within socio-cultural and experiential contexts. For instance, Michele Sambin’s \textit{videoloop} (Appendix~\ref{ax:a-michele_sambin_videoloop}) demonstrates how real-time video technology can transform linear time into circular time, turning a monologue into a dialogue between potentially infinite entities. The work starts with the intrinsic capabilities of the video device and expands into heightened experiences of its everyday functionality. The familiar LCD (or CRT) screen, commonly used daily for work, meetings, movies, and news, is reimagined on stage to distort time and create a self-dialogue. This transformation highlights how the same screen, with unchanged affordances, can evoke entirely new meanings, highlighting the hidden expressive and interactive potentials of ubiquitous technologies. Similarly, \textit{Il caos delle sfere} (Appendix~\ref{ax:b-hybrid_reactivation_il_caos_delle_sfere}) and \textit{The Hole in Space} employ ``current'' and popular technologies to create innovative experiential contexts. These works merge aesthetic and social functions, inviting audiences to rethink familiar devices and their cultural implications. Another example is Patricia Cadavid’s instruments (Appendix~\ref{ax:d-sustainability_and_longevity_of_nimes}), which emerge from the intersection of two ecosystems: traditional Colombian cultural heritage and the experimental context of New Interfaces for Musical Expression (NIME). These instruments depend on both ecosystems and evolve through interaction, blending cultural affordances with expressive potential in a contemporary framework.\\
These examples underline a critical point: artworks cannot be separated from their aesthetic or social functions. Conservation must account for the ecosystems in which the artwork operates, ensuring its capacity to evolve alongside socio-cultural transformations. This approach allows the artwork's aesthetic and social functions to survive and grow.\\
For this reason, the artwork's identity cannot remain static and detached from conservation practices, whether carried out by conservators, artists, or both. The artwork must be continuously rebuilt and reinterpreted—not only in its components but in its functionality. The \textit{ecological turn} in conservation practices, as discussed in Chapter~\ref{ch:2-new_conservation_paradigms}, emphasises the idea of the artwork as a living organism with its own \textit{biography}. Its life unfolds through the negotiation of interactions among various ecosystems—technological, institutional, creative, and, in general, socio-cultural—during reactivations.\\
Documentation is the main instrument used to support conservation and, thus, the artwork unfolding. As seen in Chapter~\ref{ch:2-new_conservation_paradigms}, documentation allows us to trace an artwork’s \textit{biography}, negotiate interactions between ecosystems, and understand its \textit{variability} and \textit{changeability}. It describes technologies, devices, and elements within the artwork, their origins, and their roles. It captures the artist’s vision, collaborators’ contributions, and the forming identity of the work across iterations. Documentation also forms the foundation for innovative conservation methodologies, as explored in Chapter~\ref{ch:1-state_of_the_art}. Over the past three decades, international initiatives have developed tools such as interviews, questionnaires, instruction modules, thesauri, and identity-definition frameworks. Among these, tracing the identity and iterations of an artwork through structured documentation templates has emerged as a critical, innovative aspect. However, despite progress in this research field, a fully standardised approach, with its benefits, remains elusive, and it may never be possible. Conservation practices and artworks remain deeply tied to the institutions and ecosystems in which they are embedded.\\
Based on this awareness, the primary contribution of this research is the \textit{Multilevel Dynamic Conservation} (MDC) model. The goal is not to pursue standardisation but to promote the heterogeneity of conservation applications, emphasising flexibility and adaptability. The model functions as a meta-model, where the prefix ``meta'' denotes an abstraction of conservation paradigms and models for live media art. Its purpose is to define the hierarchical, multilevel properties of live media art’s creative works within the context of their conservation, including the \textit{multiple belongingness} (connections between and within iterations) and the \textit{dependencies} of various levels of the work. \\
This framework aims to establish guidelines for both applying and describing the conservation of live media art. It provides a versatile tool for a wide range of entities involved in the creation and preservation of this art form—from the development of tools and technologies to the creation of performance environments and the archiving of works in museums and other institutions. The MDC model offers a methodology to create exclusive, context-specific models, fostering collaboration across diverse applications, tools, and expertise. It emphasises common elements for interaction and exchange across sectors and ecosystems, thereby promoting the \textit{ecological turn} at the archival level –through \textit{dependencies}– and, most importantly, at the practical level.\\
Significantly, to move away from standardisation, this model should not be seen as limited to documentation. While it was applied mainly to archival contexts during this research (as discussed in Chapter~\ref{ch:4-madc_model_application} and demonstrated in the case studies), its scope extends beyond these applications. Drawing from the concept of the \textit{multi-typological} archive in Chapter~\ref{ch:3-mdc_model-reactivation_workflow-instruction_template}, the model avoids categorising elements into exclusive typologies. Instead, it embraces the conservation and archival elements as abstractions–whether they are digital or physical documents, hardware or software components, materials, consumable objects, etc.–focusing on how they interact between them and across the iterations and, thus, how they contribute to the evolving identity of the artwork.\\
It is also essential to note that the variability of the artwork is not an intrinsic property of the model but rather predicted and derived through its application. For instance, the concept of \textit{multiple belongingness} across iterations can also be used to affirm an artwork's immutability. If all elements remain unchanged across iterations, connected by their \textit{multiple belongingness}, the artwork's identity and authenticity can be considered fixed. This traceability of changes can document experiential elements or even track factors of degradation and wear.\\
Therefore, although the model draws on the state-of-the-art (analysis in Chapter~\ref{ch:1-state_of_the_art}) and the conceptual framework of conservation (outlined in Chapter~\ref{ch:2-new_conservation_paradigms}) of live media art, it is not exclusively tied to these contexts. The MDC model aims to promotes openness, flexibility, and adaptability to other perspectives and conceptual frameworks, whether traditional, historical, future-oriented, or merely different.\\
In addition, this research introduces two complementary tools specifically designed for the conservation of live media art: the CATTA reactivation workflow and the \textit{Modular Instruction Framework} (MIF).\\
Grounded in the central role of documentation and the conservation concept as a loop between documentation and reactivation, the CATTA workflow offers a systematic process for live media art's present and future reactivation. The workflow consists of five sequential steps: \textit{Collection}, \textit{Assessment}, \textit{Transcription}, \textit{Transmission}, and \textit{Archiving}, from which its name is derived. This structured sequence, which is hierarchical, based on dependencies, and not rigidly discrete, reflects the interconnected activities involved in reactivating the artwork. Each step has specific goals and outcomes. As defined by the new conservation paradigms, the final step, \textit{Archiving}, closes the loop between documentation and reactivation, enabling subsequent reactivations and ensuring the artwork remains alive.\\
The last contribution, the MIF, has been developed to address the need for comprehensive technical instructions and detailed information to facilitate the reconstruction and survival of artworks. Drawing on the analogy of instructions in live media art with musical scores—commonly discussed in literature—the MIF builds on case studies of live electronic music scores (Appendix~\ref{ax:c-the_score_in_live_electronics_music}) and integrates elements from \textit{Baalman’s Visual Language} (BVL), a system recently developed by Marija Baalman for multimedia performance documentation. The MIF adopts a modular structure, allowing instructions to be created textually or visually using the BVL. These modules aim to produce a layered description of the artwork for reactivation, encompassing components, parameters, and mappings. The mappings include sublevels such as conceptual, physical, and processual (as defined in the BVL), as well as spatial and temporal aspects. This modular design ensures flexibility and adaptability. Conservators, artists, and collaborators can choose the level of detail for technical instructions, as the modules are optional and can be extended to describe low-level processes if needed. Furthermore, the modularity of the MIF allows for the adaptation of individual modules over the course of the artwork’s iterations and transformations.\\
\newline
These tools—the MDC model, the CATTA reactivation workflow, and the MIF—represent the main contributions of this research project, forming an original system for documenting, reactivating, and conserving live media art. Within these models lies the practical answer to the question posed at the beginning of this text. However, we want to address this question and conclude the text using a well-known paradox in this field of research: the \textit{Ship of Theseus} paradox. This paradox raises the metaphysical question of identity for objects whose parts change over time. At its core is the question: if every part of an object is replaced, does it remain the same object? In the Greek myth, as Theseus and the Athenians sailed from Crete, they gradually replaced the wooden parts of their ship as they deteriorated. All the original parts had been substituted by the time they reached Athens. The paradox asks: is the ship that arrived in Athens the same one that departed from Crete? Through the lens of new conservation paradigms, we shift focus from what the ship was at the start or end of its journey to its \textit{function}: carrying the Athenians across the sea from Crete to Athens. The ship’s parts were replaced as needed to conserve the \textit{function}. In the \textit{ecological turn}, the artwork is seen amidst the ``stormy seas'' of technological and socio-cultural transformations, supported and conserved by its ``crew''—artists, conservators, curators, collaborators, and performers—to safeguard its \textit{function}–aesthetic and social functions. Thus, the paradox should not be viewed as a question of an artwork’s identity but as an illustration of how to keep live media art alive. However, in practice, this metaphor is limited. The variables at play—such as the functions of the artwork, the journeys it undertakes, the challenges it faces, and the ``crew'' who maintain it—differ greatly for each piece. Artworks are heterogeneous, made of diverse elements linked to multiple ecosystems. Their creation processes, as well as the entities responsible for their exhibition and conservation, are equally heterogeneous. If conservation has become about safeguarding or ``keeping alive'' the artwork and its functions, the key principles to achieve this in practice are \textit{flexibility} and \textit{adaptability}. These are the key principles at the base of the systems introduced, which are designed and structured to ensure their applicability in several heterogeneous contexts and meet different needs. 












%!TEX root = ../main.tex

\begin{abstract}
This text explores the new paradigms and conservation practices for live media art, which combines real-time, interactive, or performative elements (live) with various media and technological devices (media). These practices particularly challenge traditional principles, such as originality and authenticity.\\
The main objective of this research was to develop and apply a meta-model for conservation, the Multilevel Dynamic Conservation (MDC) model. This model aims to surpass the very definition of a model (as suggested by the prefix “meta”) and address the challenges of standardising conservation practices. It is designed to support the description and development of multiple conservation models necessary for handling the heterogeneity of live media works and responding to the diversity of entities involved in their preservation. To support this model, an approach for reactivation (the CARIA process) and a visual language for documenting technological devices and their processes were also defined. \\
The outcomes of this research were developed through a review of online resources, literature analysis, and, especially, a series of diverse case studies. These case studies explored not only the documentation and reactivation of artworks but also the role of artists in technological development, the interdisciplinary collaboration systems that support the artwork’s creation, the technological basis of performances and installations, and the role of audiences in interactive ecosystems.\\
The application of the meta-model, reactivation process, and visual language in the case studies demonstrated strong adaptability, enabling support for works in various performative, installation, and research contexts. In particular, the MDC model proved to be highly flexible in multiple circumstances and across different computational systems, such as documenting an artwork using the graphical database Neo4j, supporting collaboration in an artistic interdisciplinary research project through the development platform GitHub, and creating an “actor archive”—personal documentation of a performer—by locally managing the documentation on a computer (in our case, Finder on macOS).\\
This study particularly highlights, on the one hand, the complexity of the new conservation practices for live media art, which must adapt and reformulate according to each artwork; on the other hand, it emphasises the fundamental principles of the new conservation paradigms upon which the models are based. The new conservation paradigms and the models developed do not aim to freeze the artwork in its original state but rather to allow it to evolve and grow. The artwork itself is seen as a living organism or process that interacts with the exhibition or performance ecosystem. This ecosystem includes media and technological devices, spaces, humans as both audience and performers, and the socio-cultural moment, which is always in constant transformation. Thus, the artwork, like a living organism, needs to adapt, grow, take shape, and develop its identity in relation to the evolving exhibition ecosystems. Therefore, the purpose of conservation, as well as the models presented in this text, is to keep live media art alive, supporting its development and becoming.
\end{abstract}
